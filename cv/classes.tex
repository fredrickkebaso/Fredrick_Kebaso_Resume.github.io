%-------------------------------------------------------------------------------
%	SECTION TITLE
%-------------------------------------------------------------------------------
\newpage
\cvsection{Relevant Training \& Short Courses}


%-------------------------------------------------------------------------------
%	CONTENT
%-------------------------------------------------------------------------------
\begin{cventries}
 
%---------------------------------------------------------
   \cventry
    {WorldQuant University} % Organisation
    {Data Science and Machine Learning } % Project
    {Online} % Location
    { July 2022 - Current} % Date(s)
    {
      \begin{cvitems} % Description(s) of project
        \item {Projects undertaken: Five of the total Eight projects completed so far}
        \item {Skills learned: Data wrangling, Model development, Model testing and Deployment}
      \end{cvitems}
    }

    %---------------------------------------------------------
    
    \cventry
    {The University of Chicago} % Organisation
    {Biological Data Science Summer Workshop} % Project
    {Virtual} % Location
    {8th Aug 2022 -- 19th Aug 2022} % Date(s)
    {
      \begin{itemize} % Description(s) of project
        \item \textbf{Topics:} Classical machine and deep learning theory.
        \item \textbf{Description:} Got introduced to the basics of classical machine learning theory, including feature selection, supervised learning, neural networks, support vector machines, and decision trees. Used Matlab to implement classification and clustering. In deep learning, I learned how to use deep machine learning to analyze protein sequence data.
      \end{itemize}
    }
    
    %-------------------------------------------------
    \cventry
    {International Centre of Insect Physiology and Ecology (ICIPE)} % Organisation
    {Bioinformatics Residential Training Course} % Project
    {Kenya} % Location
    {July 2022 -- Aug 2022} % Date(s)
    {
      \begin{itemize} % Description(s) of project
        \item \textbf{Projects:} ONTmetacriptom-NF.
        \item \textbf{Description:} A Nextflow Pipeline for Oxford Nanopore Technology (ONT) Long Reads Metatranscriptomic Data Analysis.
      \end{itemize}
    }
    
    %---------------------------------------------------------
    \cventry
    {Pwani University} % Organisation
    {Leadership in Africa Training} % Project
    {Kenya} % Location
    {June 2022 -- July 2022} % Date(s)
    {
      \begin{itemize} % Description(s) of project
        \item Topics: Leadership skills, Project management, Mentorship.
      \end{itemize}
    }
    
    %---------------------------------------------------------
    \cventry
    {H3ABioNet is a Pan African Bioinformatics Network for the Human Heredity and Health in Africa (H3Africa) consortium} % Organisation
    {Research Data Management (RDM)} % Project
    {Kenya} % Location
    {July 2022 -- Aug 2022} % Date(s)
    {
      \begin{itemize} % Description(s) of project
        \item \textbf{Description:} Trained on the principles and practices of Research Data Management (RDM). These include data discovery and re-use, data documentation and organization, data standards and Ontology, data storage and security, repositories and policies, FAIR \& reproducibility, and best practices in developing an effective Data Management Plan.
      \end{itemize}
    }

%---------------------------------------------------------

    \cventry
    {H3ABioNet is a Pan African Bioinformatics Network for the Human Heredity and Health in Africa (H3Africa) consortium} % Organisation
    { Next Generation Sequencing (NGS) } % Project
    {Virtual} % Location
    {April 2021 - June 2021} % Date(s)
    {
      \begin{itemize} % Description(s) of project
        \textbf{Description:} {The course enriched me with skills for NGS data analysis such as RNASeq, variant calling, Genome Assembly, Genome annotation, Chip Seq and General Bioinformatics algorithms such as Fastqc, Trimmomatic, Salmon, Samtools etc and Technologies such as Illumina, PacBio and Nanopore sequencing platforms}
      \end{itemize}
    }
    
%---------------------------------------------------------
\end{cventries}
